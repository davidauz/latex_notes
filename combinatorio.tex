\section{Calcolo combinatorio}\label{subsec:calcolo_combinatorio}

\subsection{Formule}
\setcounter{equation}{0}

\begin{enumerate}

\item 
Coefficiente binomiale (binomial coefficient)

\begin{equation}
\binom{n}{k} = \frac{
n\times (n-1)\times\cdots\times(n-k+1)
}{
k\times(k-1)\times\cdots\times1
}
=\frac{n!}{k!(n-k)!}
\end{equation}



per $n,k\in \mathbb{N} ,0\leq k\leq n$

L'espressione $\tbinom{n}{k}$ si legge ``\emph{n su k}"; in inglese ``\emph{n choose k}", perché la definizione del coefficiente binomiale consiste nel numero di possibili modi di scegliere (\emph{choose}) un sottoinsieme non ordinato di $k$ elementi da un insieme di $n$ elementi. 

\item
Binomial formula (o binomial identity):

\begin{equation*}
(x+y)^n=
\binom{n}{0}x^{n}y^0+
\binom{n}{1}x^{n-1}y^1+
\binom{n}{2}x^{n-2}y^2+\cdots+
\binom{n}{n-1}x^{1}y^{n-1}+
\binom{n}{n}x^{0}y^{n}
\end{equation*}

La stessa formula usando la sommatoria:

\begin{equation}
(x+y)^n=\sum_{k=0}^{n}{\binom{n}{k}x^{k}y^{n-k}}
=\sum_{k=0}^{n}{\binom{n}{k}x^{n-k}y^{k}}
\end{equation}


\item
Formula di Pascal (regola di Pascal)

\begin{equation}\label{formula_pascal}
{n \choose k}={n-1 \choose k}+{n-1 \choose k-1}
\end{equation}

\end{enumerate}

\subsection{Esercizi}

\begin{enumerate}
\item

Trovare le soluzioni di

\begin{equation*}
\binom{n}{8}=\binom{n}{6} \hspace{2cm} n \geq 8
\end{equation*}
( Soluzione a pagina \pageref{combs_01} \label{combl_01} )



\item 

\begin{equation*}
\frac{n!}{(n-3)!} = 210
\end{equation*}

( Soluzione a pagina \pageref{combs_02} \label{combl_02} )

\item
Dimostrare che 

\begin{equation*}
\sum_{k=0}^{n}{\binom{n}{k}}=2^n
\end{equation*}
( Soluzione a pagina \pageref{combs_03} \label{combl_03} )
\end{enumerate}
