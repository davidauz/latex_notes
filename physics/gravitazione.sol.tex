
\subsection*{Soluzioni di esercizi nella sezione ``\textbf{\nameref{gravitazione}}".}

Soluzione dell'esercizio \ref{grav_01} a pagina \pageref{grav_01}\label{grav_s_01}

Sappiamo che la formula che descrive la velocità di un satellite in orbita circolare è 
\[v=\sqrt{GM\over r}\]
da cui
\[v=\sqrt{6,67\times 10^{-11}\: N\cdot m^2/kg^2\times 5,97\times 10^{24}\; kg \over 600\times 10^3\:m+6,37\times 10^6\:m}\approx 7,56\times 10^3\;m/s\]

Per calcolare la velocità angolare ci ricordiamo che la relazione che lega la velocità angolare e la velocità tangenziale è

\[\omega={v\over r}={7,56\times 10^3\;m/s \over 600\times 10^3\:m+6,37\times 10^6\:m }\approx 1,08\times 10^{-3}\:rad/s\]

Infine calcoliamo il periodo di rivoluzione
\[T={2\pi r\over v}={2\pi\cdot (600\times 10^3\:m+6,37\times 10^6\:m)\over 7,56\times 10^3\;m/s }\approx 5,8\times 10^3\:s\]



Soluzione dell'esercizio \ref{grav_02} a pagina \pageref{grav_02}\label{grav_s_02}


Sappiamo che la Terra ha un periodo di rotazione di \[T_T=23,9\:h\]
se vogliamo che il satellite sia sempre sulla verticale dello stesso punto necessitiamo che i due periodi siano uguale, ossia 

\[T_s=T_T\]

\[{2\pi\cdot r_s\over v_s}=T_T\]

\[{2\pi\cdot r_s\over \sqrt{GM_T/r_s}}=T_T\]

\[{2\pi\cdot \sqrt {r_s^3}\over \sqrt{GM_T}}=T_T\]

\[\sqrt {r_s^3}={\sqrt{GM_T}\cdot T_T\over 2\pi}\]

\[r_s=\sqrt[3]{GM_T\cdot T_T^2\over 4\pi^2}\]

Per terminare ci basta ricavare la massa terrestre. Sappiamo che


\[T_L={2\pi\cdot R_L\over v_L}={2\pi\cdot r_L\over \sqrt{GM_T/r_L}}\Rightarrow M_T={4\pi^2R_L^3\over GT_L^2}\]

da cui, sostituendo nella formula scritta sopra

\[r_s=\sqrt[3]{GM_T\cdot T_T^2\over 4\pi^2}=\sqrt[3]{G\cdot 4\pi^2\cdot R_L^3\cdot T_T^2\over 4\pi^2\cdot GT_L^2}=\sqrt[3]{ R_L^3\cdot T_T^2\over T_L^2}\]

e pertanto

\[r_s=\sqrt[3]{ (3,84\times 10^8\:m)^3\cdot (23,9\cdot 3600\:s)^2\over (2,36\times 10^6\:s)^2}\approx 4,22\times 10^7\: m\]

Pertanto l’altezza a cui dovrà stazionare il satellite sarà


\[h_s=r_s-R_T=4,22\times 10^7\:m-6,37\times 10^6\:m\approx 3,58\times 10^7\:m\]


