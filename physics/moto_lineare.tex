\subsection{Moto lineare} \label{sec:motolineare}
\footnote {A.K.A. \em{Dinamica del punto}}

\setcounter{equation}{0}
\begin{enumerate}

\item A speedboat has an acceleration of 2 $\frac{m}{s^2}$.  \label{ex_f_1} 

What would the final velocity of the speedboat be after 5 seconds if the initial velocity of the speedboat is 4 $\frac{m}{s}$?

( Soluzione a pagina \pageref{sol_f_1} )

\item A vehicle has an acceleration of 5 $\frac{m}{s^2}$. \label{ex_f_2} 

What would the final velocity of the vehicle be after 10 seconds if the initial velocity of the vehicle is 20 $\frac{m}{s}$?

What is the displacement of the vehicle during the 10 $s$ time interval?

( Soluzione a pagina \pageref{sol_f_2} )








\item A child on a toboggan starts from rest and accelerates down a snow-covered hill at 0.8$\frac{m}{s^2}$.  \label{ex_f_3} 

How long does it take the child to reach the bottom of the hill if it is 25.0 m away?

( Soluzione a pagina \pageref{sol_f_3} )







\item  A car accelerates uniformly from a velocity of 21.8 $\frac{m}{s}$ to a velocity of 27.6 $\frac{m}{s}$. \label{ex_f_4} 

The car travels 36.5 m during this acceleration.


What was the acceleration of the car?


Determine the time interval over which this acceleration occurred.


( Soluzione a pagina \pageref{sol_f_4} )


\item A motorcycle starts from rest and accelerates at +2.50 $\frac{m}{s^2}$ for a distance of 150.0 m. \label{ex_f_5} 



It then slows down with an acceleration of –1.50 $\frac{m}{s^2}$ until the velocity is +10.0 m/s.

Determine the total displacement of the motorcycle.


( Soluzione a pagina \pageref{sol_f_5} )



\item 
\label{ex_f_6} 

Sulla superficie di un pianeta che ha lo stesso raggio della Terra e massa doppia di quella della Terra, un pendolo semplice, che sulla Terra compie piccole oscillazioni con un periodo di 1 s, oscillerebbe con periodo pari a:
\begin{itemize}
\item[A] $1 s$
\item[B] $2 s$
\item[C] $0.5 s$
\item[D] $\sqrt{2} s$
\item[E] $\frac{1}{\sqrt{2}} s$
\end{itemize}


( Soluzione a pagina \pageref{sol_f_6} )


\item 
\label{ex_f_7} 
Due proiettili di masse diverse vengono sparati dalla stessa altezza orizzontalmente.

La velocità iniziale, che ha quindi solo la componente orizzontale, è differente per i
due proiettili.

Trascurando ogni attrito, quale dei due proiettili impiega più tempo per arrivare a terra?

\begin{itemize}
\item[A] Entrambi impiegano lo stesso tempo
\item[B] Il proiettile sparato con velocità iniziale maggiore
\item[C] Il proiettile sparato con velocità iniziale minore
\item[D] Il proiettile con massa maggiore
\item[E] Il proiettile con massa minore
\end{itemize}



( Soluzione a pagina \pageref{sol_f_7} )

% \item 
% \label{ex_f_5} 
% 
% 
% ( Soluzione a pagina \pageref{sol_f_5} )






\end{enumerate}



