\section{Soluzioni}

\subsection*{Soluzioni di esercizi nella sezione ``\textbf{\nameref{sec:motolineare}}".}

Soluzione dell'esercizio \ref{ex_f_1} a pagina \pageref{ex_f_1}\label{sol_f_1}

\begin{itemize}
\item[$a$] = $2\frac{m}{s^2}$
\item[$u$] = $4\frac{m}{s}$
\item[$t$] = $5s$
\end{itemize}


\begin{equation}
\textrm{Final velocity}\hspace{1cm}
v = u + at = 4 + 2\cdot 5 = 14 \frac{m}{s}
\end{equation}


\vspace{1cm}
\hrule
\vspace{1cm}

Soluzione dell'esercizio \ref{ex_f_2} a pagina \pageref{ex_f_2}\label{sol_f_2}

\begin{itemize}
\item $a$ = $5 \frac{m}{s^2}$
\item $u$ = $20 \frac{m}{s}$
\item $t$ = 10 $s$
\end{itemize}


\begin{equation}
\textrm{Final velocity}\hspace{1cm}
v = u + at = 20 + 5\cdot 10 = 70 \frac{m}{s}
\end{equation}

Displacement during the 10 s time interval $s$

\begin{equation}
s = ut + \frac{1}{2}at^2
= 20 \cdot 10 + \frac{1}{2}\cdot 5 \cdot 10^2
= 200 + 250 = 450 m
\end{equation}



\vspace{1cm}
\hrule
\vspace{1cm}


Soluzione dell'esercizio \ref{ex_f_3} a pagina \pageref{ex_f_3}\label{sol_f_3}

\begin{itemize}
\item initial velocity u = 0
\item acceleration a = 0.8 $\frac{m}{s}$
\item distance = s = 25 m
\end{itemize}

\begin{equation}
\begin{split}
& s = ut +\frac{1}{2}at^2 \textrm{ \hspace{1cm}$u = 0$, so we can write:} \\
& s = \frac{1}{2}at^2\\
& \Rightarrow t^2 =(2 \frac{s}{a} ) = \frac{2×25}{0.8} = 62.5 \\
& \textrm{time }t = \sqrt{62.5} =  7.9 s
\end{split}
\end{equation}

\vspace{1cm}
\hrule
\vspace{1cm}

Soluzione dell'esercizio \ref{ex_f_4} a pagina \pageref{ex_f_4}\label{sol_f_4}

What was the acceleration of the car?


\begin{itemize}
\item $v_{final} = 27.6 \frac{m}{s}$
\item $v_{initial} = 21.8 \frac{m}{s}$
\item distance traveled $s=36.5 m$
\end{itemize}

This equation must be used: 

\begin{equation}
v_{final}^2 = v_{initial}^2 + 2as
\end{equation}

\begin{equation}
\begin{split}
& \textrm{acceleration}\hspace{0.5cm}a = \frac{v_{final}^2 – v_{initial}^2}{2s} = \\
\\
& \frac{27.62 – 21.82}{2\cdot 36.5} = 3.92 \frac{m}{s^2} \\
\end{split}
\end{equation}



Determine the time interval over which this acceleration occurred.


\begin{equation}
\begin{split}
&v_{final} = v_{initial} + at \\
\\
&t = \frac{v_{final}-v_{initial}}{a} = \frac{27.6-21.8}{3.92} \\
\\
&= 1.48 s
\end{split}
\end{equation}


\vspace{1cm}
\hrule
\vspace{1cm}

Soluzione dell'esercizio \ref{ex_f_5} a pagina \pageref{ex_f_5}\label{sol_f_5}


The first part displacement is $s = 150 m$

We need to find out the final velocity($v_{1f}$) of this accelerating part.

\begin{equation}
\begin{split}
&v_{1f}^2 = v_{1i}^2 + 2as\\
\\
\Rightarrow & v_{1f}^2 = 0 + 2\cdot 2.5 \cdot 150 = 750\\
\\
\Rightarrow & v_{1f} = \sqrt{750} = 27.4 m/s
\end{split}
\end{equation}

This velocity becomes the initial velocity of the second part of the journey.

Second part:

\begin{itemize}
\item $v_{2i} = v_{1f} = 27.4 \frac{m}{s}$
\item $a = -1.5 \frac{m}{s^2}$
\item $v_{2f}= 10 \frac{m}{s}$
\end{itemize}

\begin{equation}
\begin{split}
&v_{2f}^2 = v_{2i}^2 + 2as \\
\\
&10^2 = {27.4}^2 + 2\cdot -1.5 \cdot s \\
\\
\Rightarrow &100 = 750 – 3s \\
\\
\Rightarrow &\frac{100 -750}{-3} = s \\
\\
&s= \frac{650}{3} = 216.7 m
\end{split}
\end{equation}

Second part displacement = 216.7 m

Considering a straight-line journey without a change of direction,
total displacement = (150 + 216.7) m = 366.7 m


\subsection*{Soluzioni di esercizi nella sezione ``\textbf{\nameref{sec:powoama}}".}

\vspace{1cm}
\hrule
\vspace{1cm}

Soluzione dell'esercizio \ref{ex_f_6} a pagina \pageref{ex_f_6}\label{sol_f_6}

\begin{enumerate}
\item In the International System of Units (SI) the unit of force is the \textbf{N}ewton, equal to mass per acceleration.

\begin{equation}
Newton=ma=kg*\frac{m}{s^2}
\end{equation}

\item The unit for measuring work is the \textbf{J}oule, also the unit of energy in general, equal to the amount of work done when a force of 1 \textbf{N}ewton moves a mass of 1 Kg through a distance of 1 \textbf{m}eter. 

\begin{equation}
Joule=\textrm{Newton}*\textrm{distance}=kg*\frac{m}{s^2}*m=\frac{kg*m^2}{s^2}
\end{equation}

\item The measure for power is the \textbf{W}att, that is work divided by time, equal to 1 \textbf{J}oule per \textbf{s}econd.

\begin{equation}
\label{piw}
\textrm{Power in Watt}=\frac{\textrm{work (or energy)}}{\textrm{time}}=\frac{J}{s}=\frac{kg*m^2}{s^2}*\frac{1}{s}=\frac{kg*m^2}{s^3}
\end{equation}

% Work can also be expressed as force multiplied by distance:
% 
% \begin{equation}
% W=\frac{kg*m^2}{s^3}=m*\frac{kg*m}{s^2}=m*N\textrm{\hspace{0.5cm} (meters per Newtons)}
% \end{equation}
% 
% \item So
% 
% \begin{equation}
% \textrm{Power }P=\frac{\textrm{work}}{\textrm{time}}=\frac{J}{s}
% \end{equation}
% 
But $\frac{m}{s}$ is velocity, so

\begin{equation}
\textrm{Power }P=\frac{kg*m^2}{s^3}=\frac{kg*m}{s^2}*\frac{m}{s}=N*\textrm{velocity}
\end{equation}


Hence

\begin{equation}
\label{pfv}
P=Nv
\end{equation}

\item The force in Newton ( $N$ ) in the above equation \ref{pfv} is \textbf{m}ass multiplied by \textbf{a}cceleration.

\begin{equation}
\label{fma}
F=ma
\end{equation}


\item Acceleration is the change in velocity over the change of time.

\begin{equation}
a=\frac{\Delta v}{\Delta t}=\frac{\textrm{0 to 180 km/1h}}{\textrm{0 to 8s}}=\frac{\textrm{180*1000 m/3600 s}}{8s}=6.25\frac{m}{s^2}
\end{equation}

\item This value can be used in equation \ref{fma} together with the mass of the car to calculate the force:

\begin{equation}
F=ma=1200Kg*6.25m/s^2=7500 N
\end{equation}

\item The question is about average power so the average velocity has to be calculated:

\begin{equation}
v_\textrm{average}=\frac{\Delta v}{2}=\frac{v_\textrm{final}-v_\textrm{initial}}{2}=\frac{180-0}{2}=90\frac{km}{h}=25\frac{m}{s}
\end{equation}

\item Now for the final answer according to equation \ref{pfv}

\begin{equation}
P_\textrm{average}=Nv_\textrm{average}=7500 N * 25\frac{m}{s}=187500 W
\end{equation}

\end{enumerate}

The same problem can be also solved considering the change in Kinetic Energy.

Equation \ref{piw} says that power is work divided by time.

Work is also difference in kinetic energy.

\begin{equation}
P=\frac{W}{t}=\frac{\Delta K}{t}
\end{equation}

\begin{equation}
\textrm{Kinetic Energy\hspace{0.5cm}} K=\frac{1}{2}mv^2
\end{equation}

The Kinetic Energy of the car at the beginning ( $v_{initial}$ ) is zero, so:

\begin{equation}
\textrm{difference\hspace{0.5cm}}\Delta K=\frac{1}{2}mv_\textrm{final}^2-\frac{1}{2}mv_\textrm{initial}^2=\frac{1}{2}mv_\textrm{final}^2
\end{equation}

\begin{equation}
P=\frac{1}{2}mv_\textrm{final}^2*\frac{1}{t}=
\frac{1}{2}1800*50^2\frac{1}{8}=187500 W
\end{equation}



Soluzione dell'esercizio \ref{ex_f_7} a pagina \pageref{ex_f_7}\label{sol_f_7}

Per risolvere questo esercizio basta applicare la formula di pressione, infatti

\[p={F\over A}={7,8\: N\over 0,060\: m^2}\approx 130\: Pa\]

Dove ricordiamo esplicitamente che il Pascal è l’unità di misura della pressione del sistema internazionale ed equivale a \[Pa={N\over m^2}={kg\over m\cdot s^2}\]


Soluzione dell'esercizio \ref{ex_f_8} a pagina \pageref{ex_f_8}\label{sol_f_8}

Per prima cosa risolviamo l’esercizio calcolando esattamente la pressione che la donna esercita sulla sabbia e la pressione che il bambino esercita sulla sabbia. 

Innanzitutto consideriamo che sia la donna che il bambino siano appoggiati con entrambi i piedi sulla sabbia, pertanto
\[p_{donna}={F\over A}={m\cdot g\over A}={54\: kg\cdot 9,81\: N/kg\over 2\cdot 0,018\: m^2}=14715\: Pa\]

\[p_{bambino}={F\over A}={m\cdot g\over A}={27\: kg\cdot 9,81\: N/kg\over 2\cdot 0,009\: m^2}=14715\: Pa\]


Quindi le due pressioni sono uguali.

Osserviamo esplicitamente che potevamo giungere alla stessa conclusione osservando che il bambino pesa esattamente la metà della donna e che la superficie dei piedi del bambino è esattamente la metà della superficie dei piedi della donna.



