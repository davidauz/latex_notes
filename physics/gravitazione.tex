
\subsection{Gravitazione} \label{gravitazione}


\setcounter{equation}{0}
\begin{enumerate}

\item \label{grav_01}

Un satellite artificiale su un’orbita circolare si trova a un’altezza $h=600\:km$
dalla superficie della Terra, il cui raggio misura $R_T=6,37\times 10^3\: km$
e la cui massa vale $M=5,97\times 10^{24}\; kg$.

Calcolare:

\begin{enumerate}
\item la velocità $v$ con la quale il satellite ruota intorno alla Terra
\item la velocità angolare $\omega$ del satellite nel suo moto intorno alla Terra
\item il periodo di rivoluzione $T$
\end{enumerate}



( Soluzione a pagina \pageref{grav_s_01} )

\item \label{grav_02}

I satelliti geostazionari sono così chiamati perché rimangono sempre sulla verticale dello stesso punto della superficie terrestre.

Determina a quale altezza si trovano rispetto alla superficie del nostro pianeta.

Per risolvere il problema hai a disposizione queste formule:

\begin{enumerate}
\item Periodo di rotazione della Luna intorno alla Terra:
	\[T_L=2,36\times 10^6\:s\]
\item Raggio dell'orbita della Luna:
	\[R_L=3,84\times 10^5\:km\]
\item Periodo di rotazione della Terra:
	\[T_T=23,9\:h\]
\item Raggio della Terra:
	\[R_T=6,37\times 10^3\:km\]
\end{enumerate}

( Soluzione a pagina \pageref{grav_s_02} )


\end{enumerate}


