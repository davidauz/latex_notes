\section{Problemi vari}
\begin{enumerate}
\item 
A function $f(x)$ is defined on $\mathbb{R}$.

We have the following known items:

\begin{enumerate}
\item $f(x-1)$ is an odd function \label{laba}
\item $f(x+2)$ is an even function \label{labb}
\item In the interval $[-1,2]$, $f(x)=ax^2+b$ for some constants $a$ and $b$ \label{labc}
\item $f(1)=0$ \label{labd}
\item $f(-4)+f(3)=-3$ \label{labe}
\end{enumerate}

\vspace{1cm}
Problem: find $f(\frac{15}{2})$.

\vspace{1cm}
\hrule
\vspace{1cm}

Solution

First of all let's define the statement given at item \ref{laba}.

A function $f(x)$ is odd when $-f(x)=f(-x)$.

This does not mean that if $f(x)$ is odd, then $f(x-1)$ is odd too, or vice versa.

For example $f(x)=x^3$ is odd but $f(x)=(x-1)^3$ is not.

In this case $(x-1)$ is a function itself that we can call $g(x)=(x-1)$.

The complete statement is that $f(g(x))$ is odd:

\begin{equation}
\begin{split}
-f(g(x))&=f(g(-x))\\
\\
\textrm{with: } g(-x)&=(-x-1)\\
\\
\textrm{So: }-f(x-1)&=f(-x-1)
\end{split}
\end{equation}

The same goes for the statement at item \ref{labb}: the fact that $f(x+2)$ is even does not mean that $f(x+2)=f(-x-2)$.

Instead we must go through the same reasoning as before:

\begin{equation}
\begin{split}
f(g(x))\textrm{ is odd means that }\\
\\
f(g(x))&=f(g(-x))\\
\\
\textrm{with: } g(x)&=(x+2)\\
\\
\textrm{and: } g(-x)&=(-x+2)\\
\\
\textrm{So: }f(x+2)&=f(-x+2)
\end{split}
\end{equation}


This said, to find the solution we must first determine the $a$ and $b$ in item \ref{labc}, then find a way to convert $\frac{15}{2}$ in a way that the function $ax^2+b$ can be used in the given interval $[-1,2]$.

Since the given odd function has a big $1$ in it, let's see what happens at $x=1$.

\begin{equation}
\begin{split}
f(x) &= ax^2 + b \textrm{ in the interval [-1, 2]} \\
\\
\textrm{substituting } x &= 1 \textrm{ gives} \\
\\
f(1) = a + b &= 0 \\
\\
a+b=0\textrm{ means } b &= -a
\end{split}
\end{equation}

Next we can do the same in item \ref{labb} [that is: $f(x+2)=f(-x+2)$ ] so for $x=1$ we have

\begin{equation}
\begin{split}
f(x+2)=f(-x+2) \\
\\
f(3)=f(1)=0
\end{split}
\end{equation}

We now have values for $f(1)$ and $f(3)$, and we know that $b=-a$.

We just need another relation between $a$ and $b$ in the interval $[ -1, 2 ]$ to build a system of equations.

Item \ref{labe} tells us that $f(-4)+f(3)=-3$, so let's try using $x=3$ in item \ref{laba}:

\begin{equation}
\begin{split}
-f(x-1)&=f(-x-1) \\
\\
-f(3-1)&=f(-3-1)= \\
\\
-f(2)&=f(-4) \\
\\
\textrm{2 is in the given interval}\\
\\
\textrm{so } f(-4)=-f(2)&=-4a-b
\end{split}
\end{equation}

We now have a system of equations:

\begin{equation}
\left\{
\begin{array}{ll}
a=-b \\
-4a-b+a+b=-3
\end{array}
\right.
\end{equation}

The system gives us 

\begin{equation}
\left\{
\begin{array}{ll}
a=1 \\
b=-1
\end{array}
\right.
\end{equation}

We can now write 

\begin{equation}
\label{equa}
f(x)=x^2-1
\end{equation}

The only issue is that $\frac{15}{2}$ is not in the given interval; we can however use items \ref{laba} and \ref{labb} to find an equivalent value.

From item \ref{labb} we can write: 

\begin{equation}
\begin{split}
f\left(\frac{15}{2}\right)=f\left(\frac{11}{2}+2\right)=f\left(-\frac{11}{2}+2\right)=f\left(-\frac{7}{2}\right)
\end{split}
\end{equation}

$-\frac{7}{2}$ is still not in the given interval; let's try with item \ref{laba}:

\begin{equation}
\begin{split}
f\left(-\frac{7}{2}\right)=f\left(\frac{5}{2}-1\right)=-f\left(\frac{5}{2}-1\right)=-f\left(\frac{3}{2}\right)
\end{split}
\end{equation}

Now we can use equation \ref{equa}:

\begin{equation}
\begin{split}
f(x)&=x^2-1 \\
\\
\Rightarrow -f\left(\frac{3}{2}\right)&=-\left[ \left(\frac{3}{2}\right)^2-1\right]\\
\\
&=-\left(\frac{9}{4}-1\right) \\
\\
&=-\frac{5}{4}
\end{split}
\end{equation}

\end{enumerate}
