\section{Calcolo combinatorio}\label{subsec:calcolo_combinatorio}

\subsection{Formule}
\setcounter{equation}{0}

\begin{enumerate}

\item 
Coefficiente binomiale (binomial coefficient)

\begin{equation}
\binom{n}{k} = \frac{
n\times (n-1)\times\cdots\times(n-k+1)
}{
k\times(k-1)\times\cdots\times1
}
=\frac{n!}{k!(n-k)!}
\end{equation}



per $n,k\in \mathbb{N} ,0\leq k\leq n$

L'espressione $\tbinom{n}{k}$ si legge ``\emph{n su k}"; in inglese ``\emph{n choose k}", perché la definizione del coefficiente binomiale consiste nel numero di possibili modi di scegliere (\emph{choose}) un sottoinsieme non ordinato di $k$ elementi da un insieme di $n$ elementi. 

\item
Binomial formula (o binomial identity):

\[
(x+y)^n=
\binom{n}{0}x^{n}y^0+
\binom{n}{1}x^{n-1}y^1+
\binom{n}{2}x^{n-2}y^2+\cdots+
\binom{n}{n-1}x^{1}y^{n-1}+
\binom{n}{n}x^{0}y^{n}
\]

La stessa formula usando la sommatoria:

\begin{equation}
(x+y)^n=\sum_{k=0}^{n}{\binom{n}{k}x^{k}y^{n-k}}
=\sum_{k=0}^{n}{\binom{n}{k}x^{n-k}y^{k}}
\end{equation}


\item
Formula di Pascal (regola di Pascal)

\begin{equation}\label{formula_pascal}
{n \choose k}={n-1 \choose k}+{n-1 \choose k-1}
\end{equation}

\end{enumerate}

\subsection{Esercizi}

\begin{enumerate}
\item

Trovare le soluzioni di

\[
\binom{n}{8}=\binom{n}{6} \hspace{2cm} n \geq 8
\]
( Soluzione a pagina \pageref{combs_01} \label{combl_01} )



\item 

\[
\frac{n!}{(n-3)!} = 210
\]

( Soluzione a pagina \pageref{combs_02} \label{combl_02} )

\item
Dimostrare che 

\[
\sum_{k=0}^{n}{\binom{n}{k}}=2^n
\]
( Soluzione a pagina \pageref{combs_03} \label{combl_03} )


% https://math.libretexts.org/Courses/Cosumnes_River_College/Math_370%3A_Precalculus/07%3A_Sequences_and_the_Binomial_Theorem/7.04%3A_The_Binomial_Theorem/7.4E%3A_Exercises
\item Semplificare le espressioni date
( Soluzioni a pagina \pageref{combs_04} \label{combl_04} )

\begin{enumerate}

\item
\[
(3!)^2
\]


\item 
\[
\frac{10!}{7!}
\]

\item 
\[
\frac{7!}{2^3 3!}
\]

\item 
\[
\frac{9!}{4!3!2!}
\]

\item 
\[
\frac{(n+1)!}{n!}, n\ge 0
\]

\item 
\[
\frac{(k-1)!}{(k+2)!}, k\ge 1
\]

\item 
\[
\binom{8}{3}
\]

\item 
\[
\binom{117}{0}
\]

\item 
\[
\binom{n}{n-2}, n \ge 2
\]

\end{enumerate}


\item

Un truffatore improvvisa un gioco di strada con 13 carte differenti tra di loro.

Ad ogni partita, il truffatore contrassegna a caso su
quattro delle 13 carte e le mischia assieme alle altre, senza che nessun 
altro possa vedere dove sono i segni.

Il giocatore paga 20 \euro \hspace{1mm}per scegliere quattro carte tra le 13 a disposizione.

Se il giocatore indovina tutte e quattro le carte contrassegnate, vince 200 \euro.

Se ne indovina tre, la vincita è di 80 \euro.

Per due carte indovinate la vincita è di 30 \euro.

Una carta indovinata, 10 \euro.

Se il giocatore non indovina nessuna carta, prende comunque 2 \euro.

Domanda: qual'è la percentuale di profitto del truffatore?


( Soluzione a pagina \pageref{combs_05} \label{combl_05} )




\end{enumerate}
