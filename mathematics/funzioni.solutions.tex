
\subsection{Soluzioni di esercizi nella sezione ``\textbf{\nameref{sec:funzioni}}"}

Soluzione dell'esercizio \ref{exf_1} a pagina \pageref{exf_1}\label{solf_1}

\[
x^2 - 6x + k > 2
\]

Trasformiamo i termini $x^2-6x$ nel quadrato di un binomio e cioè nella forma $(ax+b)^2$.

Il valore di $a$ è pari a $1$; il valore di $b$ sarà tale per cui $2ab=-6$, quindi $2b=-6$, cioè $b=-3$.

\[ (x-3)^2=x^2-6x+9 \]

L'espressione da inserire al posto di $x^2 - 6x$ è $(x-3)^2 -9$:

\[ (x-3)^2-9+k>2 \]

\[ k>11-(x-3)^2 \]

$(x-3)^2$ non può che essere positivo, quindi il suo massimo valore è zero:

\[k>11\]


\vspace{1cm}
\hrule
\vspace{1cm}


