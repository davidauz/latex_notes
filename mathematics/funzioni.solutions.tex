
\subsection{Soluzioni di esercizi nella sezione ``\textbf{\nameref{sec:funzioni}}"}

Soluzione dell'esercizio \ref{exf_1} a pagina \pageref{exf_1}\label{solf_1}

\[
x^2 - 6x + k > 2
\]

Trasformiamo i termini $x^2-6x$ nel quadrato di un binomio e cioè nella forma $(ax+b)^2$.

Il valore di $a$ è pari a $1$; il valore di $b$ sarà tale per cui $2ab=-6$, quindi $2b=-6$, cioè $b=-3$.

\[ (x-3)^2=x^2-6x+9 \]

L'espressione da inserire al posto di $x^2 - 6x$ è $(x-3)^2 -9$:

\[ (x-3)^2-9+k>2 \]

\[ k>11-(x-3)^2 \]

$(x-3)^2$ non può che essere positivo, quindi il suo massimo valore è zero:

\[k>11\]


\vspace{1cm}
\hrule
\vspace{1cm}

Soluzione dell'esercizio \ref{exf_2} a pagina \pageref{exf_2}\label{solf_2}
\vspace{0.5cm}
\begin{enumerate}
\item[a)]

L'espressione della linea ci dice che $y=mx+c$; sostituiamo questa $y$ nell'espressione
della curva ( $xy = 16$ ):
\[x(mx+c)=16\]
\[mx^2 +cx -16=0\]

Siccome la linea e la curva sono tangenti, vuol dire che quest'ultima equazione
ha una sola soluzione, quindi il determinante deve essere zero:
\[b^2-4ac=0 \rightarrow c^2+64m=0\]
\[m=\frac{-c^2}{64}\]


\item[b)]
Sostituiamo la $y$ nell'espressione della curva come nel caso precedente:
\[x(-4x+c)=16\]
\[-4x^2 +cx-16=0\]
L'enunciato specifica che ci devono essere due soluzioni quindi il determinante
$b^2-4ac$ deve essere positivo.
\[c^2-256>0\]
\[c^2>256\]
\[c>16, c<-16\]
\end{enumerate}

\vspace{1cm}
\hrule
\vspace{1cm}

