\section{Aritmetica}\label{sec:aritmetica}

\begin{enumerate}
\item \label{ari_01}

La somma

\[
2^{15} + 2^{15}
\]

è uguale a

\begin{itemize}
\item[A] $2^{30}$
\item[B] $2^{16}$
\item[C] $4^{15}$
\item[D] un numero irrazionale
\item[E] $4^{30}$
\end{itemize}

( Soluzione a pagina \pageref{aris_01} )


\vspace{1cm}
\hrule
\vspace{1cm}

\item \label{ari_02}
Quanti tra i seguenti sono numeri primi?

\[
91,  100,  231,  440,  1003
\]


\begin{itemize}
\item[A] Nessuno di quelli elencati
\item[B] Uno
\item[C] Due
\item[D] Tre
\item[E] Quattro
\end{itemize}


( Soluzione a pagina \pageref{aris_02} )

\item 
Nel sistema di numerazione ternaria (in base $3$) le tre sole cifre usate sono 0, 1 e 2.
Quindi, ad esempio, si hanno le uguaglianze seguenti (nelle quali il numero in basso ricorda la base):
\[
0_{10} = 0_3 , 1_{10} = 1_3 , 2_{10} = 2_3 ,  3_{10} = 10_3 ,  4_{10} = 11_3 , 5_{10} = 12_3
\]

Per ulteriore chiarezza, l'ultima delle uguaglianze

\[
5_{10} = 12_3
\]

si legge: ``$5$ in base $10$ è uguale a $12$ in base $3$".

Quale dei seguenti numeri è $912_{10}$ in forma ternaria?

\begin{itemize}
\item[A] 
$12101_3$
\item[B]
$20121_3$
\item[C]
$1020210_3$
\item[D]
$210212_3$
\item[E]
$1010101_3$
\end{itemize}



\end{enumerate}

