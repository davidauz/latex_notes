\section{Limiti di successioni}\label{subsec:limiti:successioni}

Definizione:

Una successione di numeri reali $(a_n)_n$ converge ad un numero reale 
$l\in \mathbb{R}$ quando per qualsiasi numero reale
$\varepsilon > 0$ esiste un indice 
$n_0\in \mathbb{N}$ tale che tutti i termini della successione con indice maggiore di 
$n_0$ hanno distanza da $l$ minore di $\varepsilon$, dove la distanza è il valore assoluto
\begin{equation*}
\left| a_n-l \right|
\end{equation*}


La stessa definizione usando il formalismo matematico diventa:

\begin{equation*}
\forall \varepsilon > 0  \exists n_0\in \mathbb{N} : \left|a_n-l\right| < \varepsilon
\forall n>n_0
\end{equation*}

Quando l'enunciato è vero si scrive

\begin{equation*}
\lim_{n\to\infty}a_n = l
\end{equation*}

e si legge così: 


Il limite per n tendente all'infinito di `a n' è `l'

\subsection{Esercizi}

\begin{enumerate}




\item  dimostrare che  \label{lims_00}
% https://www.youmath.it/domande-a-risposte/view/3114-verifica-limite.html
\begin{equation*}
\lim_{n\to + \infty}\frac{n}{2n+5}=\frac{1}{2}
\end{equation*}
\rightline{ (Soluzione a pagina \pageref{limss_00} )}


\item  dimostrare che  \label{lims_01}
% https://www.youmath.it/lezioni/analisi-matematica/successioni/554-definizione-limite-successione.html
\begin{equation*}
\lim_{n\to + \infty}\frac{n-1}{n}=1
\end{equation*}
\rightline{ (Soluzione a pagina \pageref{limss_01} )} % limits.solutions.tex

% https://www.youmath.it/esercizi/es-analisi-matematica/esercizi-successioni/1796-esercizi-risolti-sulla-verifica-del-limite-di-una-successione.html


\item Calcolare \label{lims_02}
\begin{equation*}
\lim_{n \to +\infty}\frac{2n+1}{3n-1}
\end{equation*}
\rightline{ (Soluzione a pagina \pageref{limss_02} )} % limits.solutions.tex


\end{enumerate}

\pagebreak

