

\subsection{Soluzioni di esercizi nella sezione ``\textbf{\nameref{subsec:limiti:successioni}}".}

Soluzione dell'esercizio \ref{lims_00} a pagina \pageref{lims_00}\label{limss_00}

In base alla definizione bisogna dimostrare che per un qualsiasi$\varepsilon > 0$
vale questa disuguaglianza: 

\begin{equation*}
\left|
\frac{
n
}{
2n+5
}-\frac{1}{2}
\right|
<\varepsilon
\end{equation*}


\begin{equation*}
\left|
\frac{
2n -2n -5
}{
4n+10
}
\right|
<\varepsilon
\end{equation*}


\begin{equation*}
\left|
\frac{
-5
}{
4n+10
}
\right|
<\varepsilon
\end{equation*}

Si può togliere il modulo (valore assoluto) perché i due termini della disuguaglianza sono entrambi positivi:


\begin{equation*}
\frac{
-5
}{
4n+10
}
<\varepsilon
\end{equation*}


\begin{equation*}
5
<\varepsilon
\cdot(4n+10)
\end{equation*}



\begin{equation*}
5
<4n\varepsilon+10\varepsilon
\end{equation*}

Ora si risolve per $n$

\begin{equation*}
-4n\varepsilon
<
-5
+10\varepsilon
\end{equation*}


\begin{equation*}
-n
<
\frac{
-5
+10\varepsilon
}{4\varepsilon}
\end{equation*}

ossia 

\begin{equation*}
n
>
\frac{
5
-10\varepsilon
}{4\varepsilon}
\end{equation*}

Questo significa che per qualsiasi $\varepsilon$ è possibile calcolare un numero $n$ che soddisfa la definizione, quindi la dimostrazione è fatta.

\vspace{1cm}
\hrule
\vspace{1cm}



Soluzione dell'esercizio \ref{lims_01} a pagina \pageref{lims_01}\label{limss_01}

Come sempre in questo tipo di esercizi, bisogna dimostrare che dato un $\varepsilon > 0$, esiste un $n_0$ tale per cui
\begin{equation*}
\forall n>n_0: \left| a_n - a \right| < \varepsilon
\end{equation*}

Nel caso dell'esercizio:
\begin{equation*}
a_n = \frac{n-1}{n} \textrm{\hspace{1cm}mentre \hspace{1cm}} a=1
\end{equation*}

quindi la disequazione diventa:

\begin{equation*}
\left| \frac{n-1}{n} -1 \right| < \varepsilon
\end{equation*}

ossia

\begin{equation*}
\left| \frac{n-1-n}{n} \right| < \varepsilon
\end{equation*}

\begin{equation*}
\left| \frac{1}{n} \right| < \varepsilon
\end{equation*}

Siccome $\frac{1}{n}$ è una quantità positiva, possiamo tralasciare il valore assoluto:


\begin{equation*}
\frac{1}{n} < \varepsilon \hspace{1cm} \Rightarrow  \hspace{1cm} n > \frac{1}{\varepsilon}
\end{equation*}

Concludendo, per qualsiasi $\varepsilon$ è possibile calcolare un numero $n$ tale per cui la definizione di limite è vera, quindi il limite della successione è effettivamente pari a $1$.



\vspace{1cm}
\hrule
\vspace{1cm}



Soluzione dell'esercizio \ref{lims_02} a pagina \pageref{lims_02}\label{limss_02}

\begin{equation*}
\lim_{n \to +\infty}\frac{2n+1}{3n-1}=
\end{equation*}

Raccogliere $n$

\begin{equation*}
\lim_{n \to +\infty}\frac{n
\left( 
	2+\frac{1}{n}
\right)
}{n
\left( 
	3-\frac{1}{n}
\right)
}=
\end{equation*}

Semplificare

\begin{equation*}
\lim_{n \to +\infty}\frac{
	2+\frac{1}{n}
}{
	3-\frac{1}{n}
}=
\end{equation*}

Per $n$ che tende all'infinito $\frac{1}{n}$ tende a zero, quindi


\begin{equation*}
\lim_{n \to +\infty}\frac{
	2+\frac{1}{n}
}{
	3-\frac{1}{n}
}=\frac{2}{3}
\end{equation*}

Ora come al solito bisogna usare la definizione di limite per dimostrare che 

\begin{equation*}
\lim_{n \to +\infty}\frac{2n+1}{3n-1}=\frac{2}{3}
\end{equation*}

quindi dato un $\varepsilon > 0$ trovare un $n_0$ tale che per ogni $n>n_0$ si abbia
% https://www.youmath.it/forum/tutto-sulle-funzioni-da-r-a-r-e-sui-numeri-reali-analisi-matematica/39870-verificare-il-limite-di-una-successione-fratta-con-la-definizione.html

\begin{equation*}
\left|
\frac{2n+1}{3n-1}-\frac{2}{3}
\right| < \varepsilon
\end{equation*}

\begin{equation*}
\left|
\frac{
3(2n+1)-2(3n-1)
}{
(3n-1)3
}
\right| < \varepsilon
\end{equation*}

\begin{equation*}
\left|
\frac{
\cancel{6n}+3-\cancel{6n}+2
}{
9n-3
}
\right| < \varepsilon
\end{equation*}

\begin{equation*}
\left|
\frac{
5
}{9n-3}
\right| < \varepsilon
\end{equation*}

Entrambi i tyermini sono positivi quindi non serve il valore assoluto:

\begin{equation*}
\frac{
5
}{9n-3}
< \varepsilon
\end{equation*}

Ora bisogna risolvere per $n$:

\begin{equation*}
\frac{
5
}{9n-3}
- \varepsilon <0
\end{equation*}


\begin{equation*}
\frac{5 -9n\varepsilon +3\varepsilon
}{9n-3
} < 0
\end{equation*}

Il denominatore ($9n-3$) è sempre positivo perché per definizione $n$ è positivo.

Quindi la disequazione si riduce a:

\begin{equation*}
5-9n\varepsilon +3\varepsilon < 0
\end{equation*}

\begin{equation*}
-9n\varepsilon < -5 -3\varepsilon
\end{equation*}

\begin{equation*}
n > \frac{5+3\varepsilon}{9\varepsilon}
\end{equation*}

Dato un qualsiasi $\varepsilon$ esiste quindi un numero $n_0 = \frac{5+3\varepsilon}{9\varepsilon}$ tale per cui qualsiasi $n>n_0$ soddisfa la definizione di limite: la dimostrazione è terminata.
