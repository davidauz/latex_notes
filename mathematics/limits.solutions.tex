

\subsection{Soluzioni di esercizi nella sezione ``\textbf{\nameref{subsec:limiti:successioni}}".}

Soluzione dell'esercizio \ref{lims_00} a pagina \pageref{lims_00}\label{limss_00}

In base alla definizione bisogna dimostrare che per un qualsiasi$\varepsilon > 0$
vale questa disuguaglianza: 

\[
\left|
\frac{
n
}{
2n+5
}-\frac{1}{2}
\right|
<\varepsilon
\]


\[
\left|
\frac{
2n -2n -5
}{
4n+10
}
\right|
<\varepsilon
\]


\[
\left|
\frac{
-5
}{
4n+10
}
\right|
<\varepsilon
\]

Si può togliere il modulo (valore assoluto) perché i due termini della disuguaglianza sono entrambi positivi:


\[
\frac{
-5
}{
4n+10
}
<\varepsilon
\]


\[
5
<\varepsilon
\cdot(4n+10)
\]



\[
5
<4n\varepsilon+10\varepsilon
\]

Ora si risolve per $n$

\[
-4n\varepsilon
<
-5
+10\varepsilon
\]


\[
-n
<
\frac{
-5
+10\varepsilon
}{4\varepsilon}
\]

ossia 

\[
n
>
\frac{
5
-10\varepsilon
}{4\varepsilon}
\]

Questo significa che per qualsiasi $\varepsilon$ è possibile calcolare un numero $n$ che soddisfa la definizione, quindi la dimostrazione è fatta.

\vspace{1cm}
\hrule
\vspace{1cm}



Soluzione dell'esercizio \ref{lims_01} a pagina \pageref{lims_01}\label{limss_01}

Come sempre in questo tipo di esercizi, bisogna dimostrare che dato un $\varepsilon > 0$, esiste un $n_0$ tale per cui
\[
\forall n>n_0: \left| a_n - a \right| < \varepsilon
\]

Nel caso dell'esercizio:
\[
a_n = \frac{n-1}{n} \textrm{\hspace{1cm}mentre \hspace{1cm}} a=1
\]

quindi la disequazione diventa:

\[
\left| \frac{n-1}{n} -1 \right| < \varepsilon
\]

ossia

\[
\left| \frac{n-1-n}{n} \right| < \varepsilon
\]

\[
\left| \frac{1}{n} \right| < \varepsilon
\]

Siccome $\frac{1}{n}$ è una quantità positiva, possiamo tralasciare il valore assoluto:


\[
\frac{1}{n} < \varepsilon \hspace{1cm} \Rightarrow  \hspace{1cm} n > \frac{1}{\varepsilon}
\]

Concludendo, per qualsiasi $\varepsilon$ è possibile calcolare un numero $n$ tale per cui la definizione di limite è vera, quindi il limite della successione è effettivamente pari a $1$.



\vspace{1cm}
\hrule
\vspace{1cm}



Soluzione dell'esercizio \ref{lims_02} a pagina \pageref{lims_02}\label{limss_02}

\[
\lim_{n \to +\infty}\frac{2n+1}{3n-1}=
\]

Raccogliere $n$

\[
\lim_{n \to +\infty}\frac{n
\left( 
	2+\frac{1}{n}
\right)
}{n
\left( 
	3-\frac{1}{n}
\right)
}=
\]

Semplificare

\[
\lim_{n \to +\infty}\frac{
	2+\frac{1}{n}
}{
	3-\frac{1}{n}
}=
\]

Per $n$ che tende all'infinito $\frac{1}{n}$ tende a zero, quindi


\[
\lim_{n \to +\infty}\frac{
	2+\frac{1}{n}
}{
	3-\frac{1}{n}
}=\frac{2}{3}
\]

Ora come al solito bisogna usare la definizione di limite per dimostrare che 

\[
\lim_{n \to +\infty}\frac{2n+1}{3n-1}=\frac{2}{3}
\]

quindi dato un $\varepsilon > 0$ trovare un $n_0$ tale che per ogni $n>n_0$ si abbia
% https://www.youmath.it/forum/tutto-sulle-funzioni-da-r-a-r-e-sui-numeri-reali-analisi-matematica/39870-verificare-il-limite-di-una-successione-fratta-con-la-definizione.html

\[
\left|
\frac{2n+1}{3n-1}-\frac{2}{3}
\right| < \varepsilon
\]

\[
\left|
\frac{
3(2n+1)-2(3n-1)
}{
(3n-1)3
}
\right| < \varepsilon
\]

\[
\left|
\frac{
\cancel{6n}+3-\cancel{6n}+2
}{
9n-3
}
\right| < \varepsilon
\]

\[
\left|
\frac{
5
}{9n-3}
\right| < \varepsilon
\]

Entrambi i tyermini sono positivi quindi non serve il valore assoluto:

\[
\frac{
5
}{9n-3}
< \varepsilon
\]

Ora bisogna risolvere per $n$:

\[
\frac{
5
}{9n-3}
- \varepsilon <0
\]


\[
\frac{5 -9n\varepsilon +3\varepsilon
}{9n-3
} < 0
\]

Il denominatore ($9n-3$) è sempre positivo perché per definizione $n$ è positivo.

Quindi la disequazione si riduce a:

\[
5-9n\varepsilon +3\varepsilon < 0
\]

\[
-9n\varepsilon < -5 -3\varepsilon
\]

\[
n > \frac{5+3\varepsilon}{9\varepsilon}
\]

Dato un qualsiasi $\varepsilon$ esiste quindi un numero $n_0 = \frac{5+3\varepsilon}{9\varepsilon}$ tale per cui qualsiasi $n>n_0$ soddisfa la definizione di limite: la dimostrazione è terminata.
