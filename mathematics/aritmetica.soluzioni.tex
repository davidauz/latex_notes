

\subsection{Soluzioni di esercizi nella sezione ``\textbf{\nameref{sec:aritmetica}}".}

Soluzione dell'esercizio \ref{ari_01} a pagina \pageref{ari_01}\label{aris_01}

La risposta esatta è la ``B".

\[
2^{15} + 2^{15} = 2 \times 2^{15} = 2^{16}
\]


\vspace{1cm}
\hrule
\vspace{1cm}


Soluzione dell'esercizio \ref{ari_02} a pagina \pageref{ari_02}\label{aris_02}

La risposta esatta è la ``A".

I numeri $100$ e $440$ non sono primi perché pari (divisibili per $2$).


Il numero $231$ non è primo perché è divisibile per $3$, in base al criterio di divisibilità per
3: ``se la somma delle cifre di un numero è divisibile per 3, il numero è divisibile per 3".  In questo 
caso $2 + 3 + 1 = 6$ che è divisibile per 3.

Il numero $91$ è primo?
In generale per decidere se un numero $n$ è primo occorre verificare che non sia divisibile per nessun numero 
primo compreso tra 2 e $\sqrt{n}$.

Inoltre, per i criteri di divisibilità, $91$ non è divisibile per $2$ né per $3$ o per $5$.
È è divisibile per $7$?
Sì, perché
\[
 91 = 70 + 21 = 7 \times 10 + 7 \times 3 = 7 \times 13
\]

Il numero $1003$ è primo? I criteri di divisibilità dicono che non è divisibile per $2$, $3$, $5$, $11$.
Per essere certi che sia primo, dobbiamo verificare che non sia divisibile per nessun numero
primo $\leq \sqrt{1003}$, ossia (oltre a quelli già elencati), per
\[
7, 13, 17, 19, 23, 29, 31
\]

Carta, penna e pazienza ci dicono che
\begin{itemize}
\item $\frac{1003}{7}= 143$ con resto $2$
\item $\frac{1003}{13} = 77$ con resto $2$
\item $\frac{1003}{17} = 59$ con resto $0$
\end{itemize}

Dunque 
\[
1003 = 17 \times 59
\]

Nemmeno $1003$ è primo. 


