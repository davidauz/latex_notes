\section{Funzioni} \label{sec:funzioni}

\subsection{links}


\href{https://pastpapers.papacambridge.com}{papacambridge.com}\footnote{\texttt{https://pastpapers.papacambridge.com}}


\subsection{Esercizi}

\begin{enumerate}
\item  
The function $f$ is defined for $x \in \mathbb{R}$
by 
\[
f(x) = x^2 - 6x + k
\]

where $k$ is a constant.

It is given that $f(x) > 2$ for all values of $x$.

Find the set of possible values of $k$.

\rightline{( Soluzione a pagina \pageref{solf_1} \label{exf_1} )}



\item
The equation of a line is \[ y = mx + c \] where $m$ and $c$ are constants.

The equation of a curve is \[ xy = 16 \]

\begin{enumerate}
\item Given that the line is a tangent to the curve, express $m$ in terms of $c$.
\item Given instead that $m = -4$, find the set of values of $c$ for which the line 
intersects the curve at two distinct points.
\end{enumerate}

\rightline{( Soluzione a pagina \pageref{solf_2} \label{exf_2} )}

\item
The equation of a curve is \[ y=2x^2 + kx + k - 1 \] where $k$ is a constant.

\begin{enumerate}
\item Given that the line $y=2x+3$ is tangent to the curve, find the value of $k$.
\item It is now given that $k=2$.  Express the equation of the curve in the form $y=2(x+a)^2+b$
where $a$ and $b$ are constants, and hence state the coordinates of the vertex of the curve.
\end{enumerate}

\rightline{( Soluzione a pagina \pageref{solf_3} \label{exf_3} )}

\item
Risolvi in $\mathbb{R}$: 
\[ \sqrt{x^3}+\sqrt{x^2-7}+\sqrt{x}=13 \]

\rightline{( Soluzione a pagina \pageref{solf_4} \label{exf_4} )}

\end{enumerate}


